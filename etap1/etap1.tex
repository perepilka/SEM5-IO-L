\documentclass[12pt,a4paper]{article}

% Pakiety
\usepackage[utf8]{inputenc}
\usepackage[polish]{babel}
\usepackage[T1]{fontenc}
\usepackage{geometry}
\usepackage{graphicx}
\usepackage{xcolor}
\usepackage{titlesec}
\usepackage{fancyhdr}
\usepackage{tcolorbox}
\usepackage{enumitem}
\usepackage{hyperref}
\usepackage{lmodern}
\usepackage{microtype}
\usepackage{amsmath}
\usepackage{tikz}

% Konfiguracja strony
\geometry{
    a4paper,
    left=2.5cm,
    right=2.5cm,
    top=3cm,
    bottom=3cm
}

% Kolory Politechniki Wrocławskiej
\definecolor{pwrred}{RGB}{154,52,45}      % PANTONE 484 - główny kolor PWr
\definecolor{pwrgold}{RGB}{241,209,162}   % PANTONE 156 - akcentowy kolor PWr
\definecolor{lightgray}{RGB}{245,245,245}

% Konfiguracja hyperref
\hypersetup{
    colorlinks=true,
    linkcolor=pwrred,
    urlcolor=pwrred,
    citecolor=pwrred
}

% Formatowanie sekcji
\titleformat{\section}
{\color{pwrred}\Large\bfseries}
{\thesection}{1em}{}[\color{pwrgold}\titlerule]

\titleformat{\subsection}
{\color{pwrred}\large\bfseries}
{\thesubsection}{1em}{}

\titleformat{\subsubsection}
{\color{pwrred}\normalsize\bfseries}
{\thesubsubsection}{1em}{}

% Nagłówki i stopki
\pagestyle{fancy}
\fancyhf{}
\fancyhead[L]{\small\textcolor{pwrred}{Etap 1 - Modelowanie wymagań}}
\fancyhead[R]{\small\textcolor{pwrred}{Radionenko, Perepilka}}
\fancyfoot[C]{\thepage}
\renewcommand{\headrulewidth}{0.5pt}
\renewcommand{\headrule}{\hbox to\headwidth{\color{pwrgold}\leaders\hrule height \headrulewidth\hfill}}

% Styl dla przypadków użycia
\newtcolorbox{usecase}[1]{
    colback=lightgray,
    colframe=pwrred,
    fonttitle=\bfseries,
    title=#1,
    arc=2mm,
    boxrule=1pt,
    top=8pt,
    bottom=8pt,
    left=10pt,
    right=10pt,
    before skip=10pt,
    after skip=10pt
}

\begin{document}

% === STRONA TYTUŁOWA ===
\begin{titlepage}
    % Tło - gradient z dekoracyjnymi elementami
    \begin{tikzpicture}[remember picture,overlay]
        % Główne tło PWr
        \fill[pwrred] (current page.north west) rectangle (current page.south east);
        
        % Dekoracyjne paski w kolorze złotym PWr
        \fill[pwrgold] ([xshift=-2cm,yshift=-4cm]current page.north west) 
            -- ([xshift=2cm,yshift=-4cm]current page.north east)
            -- ([xshift=2cm,yshift=-4.3cm]current page.north east)
            -- ([xshift=-2cm,yshift=-4.3cm]current page.north west) -- cycle;
        
        \fill[pwrgold] ([xshift=-2cm,yshift=-18.5cm]current page.north west) 
            -- ([xshift=2cm,yshift=-18.5cm]current page.north east)
            -- ([xshift=2cm,yshift=-18.8cm]current page.north east)
            -- ([xshift=-2cm,yshift=-18.8cm]current page.north west) -- cycle;
        
        % Dekoracyjne koła w tle
        \foreach \i in {1,...,8} {
            \pgfmathsetmacro{\xpos}{2 + \i*2.5}
            \pgfmathsetmacro{\ypos}{-2 - \i*1.2}
            \pgfmathsetmacro{\opacity}{0.03 + \i*0.01}
            \fill[white,opacity=\opacity] (\xpos,\ypos) circle (1.5cm);
        }
        
        % Dodatkowe subtelne elementy geometryczne
        \draw[white,opacity=0.1,line width=0.5mm] 
            ([xshift=1cm,yshift=-8cm]current page.north west) 
            -- ([xshift=-1cm,yshift=-12cm]current page.north east);
        \draw[white,opacity=0.1,line width=0.5mm] 
            ([xshift=1cm,yshift=-15cm]current page.north west) 
            -- ([xshift=-1cm,yshift=-17cm]current page.north east);
    \end{tikzpicture}
    
    \vspace*{1cm}
    
    \begin{center}
        % Numer etapu w eleganckim boxie
        \begin{tcolorbox}[
            colback=white,
            colframe=white,
            width=5cm,
            arc=3mm,
            boxrule=0pt,
            left=5pt,
            right=5pt,
            top=3pt,
            bottom=3pt
        ]
            \centering
            {\color{pwrred}\Large\bfseries ETAP 1}
        \end{tcolorbox}
        
        \vspace{0.3cm}
        
        % Tytuł etapu
        {\color{white}\LARGE\bfseries Modelowanie wymagań}\\[0.2cm]
        {\color{white}\Large i przypadków użycia}
        
        \vspace{1.5cm}
        
        % Główny tytuł projektu w eleganckim białym boxie
        \begin{tcolorbox}[
            colback=white,
            colframe=pwrgold,
            width=0.85\textwidth,
            arc=3mm,
            boxrule=2pt,
            top=12pt,
            bottom=12pt,
            left=15pt,
            right=15pt,
            drop shadow={opacity=0.3,shadow xshift=3pt,shadow yshift=-3pt}
        ]
            \centering
            {\color{pwrred}\Large\bfseries Zintegrowany System\\[0.25cm]
            Sprzedaży Biletów\\[0.2cm]
            dla Sieci Multikin}
        \end{tcolorbox}
        
        \vspace{2cm}
        
        % Sekcja autorów z nowoczesnymi boxami
        {\color{white}\large\bfseries Autorzy projektu:}\\[0.5cm]
        
        \begin{minipage}{0.42\textwidth}
            \begin{tcolorbox}[
                colback=white,
                colframe=white,
                arc=2mm,
                boxrule=0pt,
                top=8pt,
                bottom=8pt,
                left=10pt,
                right=10pt
            ]
                \centering
                {\color{pwrred}\large\bfseries Oleksandr Radionenko}\\[0.2cm]
                {\color{gray}\normalsize nr albumu: 274003}
            \end{tcolorbox}
        \end{minipage}
        \hfill
        \begin{minipage}{0.42\textwidth}
            \begin{tcolorbox}[
                colback=white,
                colframe=white,
                arc=2mm,
                boxrule=0pt,
                top=8pt,
                bottom=8pt,
                left=10pt,
                right=10pt
            ]
                \centering
                {\color{pwrred}\large\bfseries Yaroslav Perepilka}\\[0.2cm]
                {\color{gray}\normalsize nr albumu: 282279}
            \end{tcolorbox}
        \end{minipage}
        
        \vfill
        
        % Stopka z informacją o uczelni
        \begin{tcolorbox}[
            colback=white,
            colframe=white,
            width=0.75\textwidth,
            arc=2mm,
            boxrule=0pt,
            top=10pt,
            bottom=10pt
        ]
            \centering
            {\color{pwrred}\large\bfseries Politechnika Wrocławska}\\[0.15cm]
            {\color{gray}\normalsize Wydział Informatyki i Telekomunikacji}\\[0.1cm]
            {\color{gray}\normalsize Kierunek: Informatyka Techniczna}\\[0.3cm]
            {\color{gray}\small Przedmiot: Inżynieria Oprogramowania}\\[0.3cm]
            {\color{gray}\normalsize Wrocław, \today}
        \end{tcolorbox}
        
        \vspace{0.5cm}
    \end{center}
\end{titlepage}

\newpage
\tableofcontents
\newpage

% === TREŚĆ DOKUMENTU ===

\section{Miejsce wdrożenia}

\subsection{Nazwa projektu}
\textbf{Zintegrowany System Sprzedaży Biletów dla Sieci Multikin}, zwany dalej systemem.

\subsection{Klient}
Sieć kin „MultiKino", posiadająca 3 obiekty kinowe w różnych miastach, zwana dalej siecią kin.

\subsection{Cel projektu}
Wykonanie systemu informatycznego wspierającego sprzedaż biletów w sieci kin, zarówno w kasach stacjonarnych, jak i przez internet.

System umożliwia zarządzanie repertuarem, rezerwacjami miejsc, sprzedażą online, raportowaniem oraz centralne zarządzanie danymi o seansach i salach kinowych.

\subsection{Zasoby ludzkie}
Użytkownikami systemu są pracownicy sieci kin oraz klienci:

\begin{itemize}[leftmargin=*,itemsep=0.3em]
    \item \textbf{Kasjer} -- pracownik obsługi klienta w każdym kinie. Sprzedaje bilety w kasie, przyjmuje płatności gotówkowe i kartowe, wydaje bilety papierowe, obsługuje rezerwacje dokonane online, może też anulować błędne transakcje.
    
    \item \textbf{Menedżer Kina} -- osoba odpowiedzialna za organizację repertuaru w konkretnym kinie. Dodaje i edytuje seanse, zarządza salami kinowymi, ustala lokalne ceny i kontroluje pracę kasjerów. Może generować raporty sprzedaży w swoim kinie.
    
    \item \textbf{Administrator Sieci Kin} -- osoba odpowiedzialna za nadzór nad całą siecią kin. Zarządza centralną bazą danych, ustala globalny repertuar, nadzoruje integrację z systemem sprzedaży online, generuje zbiorcze raporty finansowe i statystyki sprzedaży dla wszystkich kin.
    
    \item \textbf{Klient} -- osoba dokonująca zakupu biletu przez stronę internetową. Może przeglądać repertuar, wybierać miejsca, kupować bilety online, pobierać bilety elektroniczne lub otrzymywać je na e-mail.
\end{itemize}

\subsection{Przepisy i strategie}

\begin{enumerate}[leftmargin=*,itemsep=0.3em]
    \item Działalność sieci kin podlega przepisom ustawy z dnia 9 listopada 2018 r. o imprezach artystycznych i rozrywkowych oraz przepisom dotyczącym sprzedaży usług elektronicznych.
    
    \item Przetwarzanie danych osobowych klientów odbywa się zgodnie z art. 13 ust. 1 i 2 Rozporządzenia Parlamentu Europejskiego i Rady (UE) 2016/679 z dnia 27 kwietnia 2016 r. (RODO).
    
    \item Administratorem danych osobowych jest Administrator Sieci Kin, działający w imieniu sieci FilmLux Group.
    
    \item System wspomaga i automatyzuje procesy sprzedaży w kasach oraz online, zapewniając spójność danych w całej sieci.
    
    \item Każda transakcja sprzedaży (zarówno lokalna, jak i internetowa) jest rejestrowana w centralnej bazie danych.
    
    \item Menedżerowie kin są zobowiązani do bieżącej aktualizacji repertuaru i kontrolowania poprawności danych wprowadzanych przez kasjerów.
    
    \item Sprzedaż online działa całodobowo i podlega zasadom bezpieczeństwa transakcji elektronicznych (szyfrowanie SSL/TLS, integracja z bramkami płatniczymi).
    
    \item System musi zapewniać zgodność z wewnętrznymi procedurami dotyczącymi zwrotów, rezerwacji i anulowania transakcji.
\end{enumerate}

\subsection{Dane techniczne}

\begin{enumerate}[leftmargin=*,itemsep=0.3em]
    \item Każde kino wyposażone jest w lokalne stanowiska kasowe (2--4 komputery) z systemem Windows 11 lub Ubuntu 24.04 LTS oraz drukarkami biletów termicznych i terminalami płatniczymi.
    
    \item System działa w architekturze \textbf{klient--serwer}, z centralną bazą danych w chmurze (serwer główny sieci kin).
    
    \item Dostęp do systemu możliwy jest przez aplikację stacjonarną (dla kasjerów i menedżerów) oraz przez stronę WWW (dla klientów).
    
    \item System obsługuje do 30 jednoczesnych użytkowników (kasjerów i menedżerów) oraz do 500 klientów online jednocześnie.
    
    \item System przechowuje dane o maksymalnie 10\,000 seansach, 30 salach kinowych i 15\,000 transakcjach miesięcznie.
    
    \item Transakcje online są realizowane przez zintegrowany moduł płatności elektronicznych (np. PayU, Przelewy24).
    
    \item Dane są zabezpieczane przez codzienne automatyczne kopie zapasowe i szyfrowane transmisje danych.
\end{enumerate}

\newpage
\section{Wymagania stawiane tworzonemu oprogramowaniu}

\subsection{Wymagania funkcjonalne}

\begin{itemize}[leftmargin=*,itemsep=0.3em]
    \item \textbf{F01)} System przechowuje i przetwarza dane o seansach filmowych, salach, biletach, użytkownikach i transakcjach w całej sieci kin.
    
    \item \textbf{F02)} Kasjer sprzedaje bilety w kasie na wybrany seans filmowy.
    
    \item \textbf{F03)} Klient kupuje bilety online przez stronę WWW, wybierając film, datę, godzinę i miejsca na sali.
    
    \item \textbf{F04)} System automatycznie blokuje miejsca podczas procesu zakupu i odblokowuje je w przypadku przerwanej transakcji.
    
    \item \textbf{F05)} Menedżer Kina zarządza repertuarem lokalnego kina (dodaje, edytuje, usuwa seanse).
    
    \item \textbf{F06)} Administrator Sieci Kin zarządza repertuarem i danymi dla wszystkich kin.
    
    \item \textbf{F07)} System automatycznie aktualizuje dostępność miejsc w czasie rzeczywistym w kasach i online.
    
    \item \textbf{F08)} System generuje raporty sprzedaży dla pojedynczego kina (dla menedżera) oraz zbiorcze raporty sieciowe (dla administratora).
    
    \item \textbf{F09)} System umożliwia anulowanie lub zwrot biletów zgodnie z regulaminem sieci kin.
    
    \item \textbf{F10)} System umożliwia logowanie użytkowników z różnymi poziomami uprawnień (kasjer, menedżer, administrator).
    
    \item \textbf{F11)} System wysyła klientowi bilety elektroniczne e-mailem po zakończeniu zakupu online.
    
    \item \textbf{F12)} System rejestruje każdą operację sprzedaży, edycji repertuaru i anulowania biletów.
\end{itemize}

\subsection{Wymagania niefunkcjonalne}

\begin{itemize}[leftmargin=*,itemsep=0.3em]
    \item \textbf{N01)} System działa w architekturze klient--serwer z centralną bazą danych dostępną dla wszystkich kin.
    
    \item \textbf{N02)} System online jest dostępny 24/7 (i obsługuje do 500 równoczesnych transakcji).
    
    \item \textbf{N03)} Wszystkie dane przesyłane między klientem a serwerem są szyfrowane (SSL/TLS).
    
    \item \textbf{N04)} System zapewnia jednoczesną pracę kasjerów i klientów online bez konfliktów w rezerwacjach miejsc.
    
    \item \textbf{N05)} Interfejs użytkownika systemu jest zgodny z zasadami ergonomii i dostępności.
    
    \item \textbf{N06)} System tworzy codzienne kopie zapasowe danych.
    
    \item \textbf{N07)} Dostęp do danych administracyjnych mają tylko użytkownicy z uprawnieniami menedżera lub administratora.
    
    \item \textbf{N08)} System spełnia wymogi RODO i ustawy o świadczeniu usług drogą elektroniczną.
    
    \item \textbf{N09)} Interfejs użytkownika (zarówno kasowy, jak i internetowy) jest intuicyjny i prosty.
    
    \item \textbf{N10)} System jest skalowalny i może obsłużyć kolejne kina bez potrzeby modyfikacji kodu źródłowego.
\end{itemize}

\section{Diagram przypadków użycia}

Diagram przedstawia głównych aktorów systemu oraz ich interakcje z poszczególnymi funkcjami. System obsługuje czterech podstawowych aktorów:

\begin{itemize}[leftmargin=*,itemsep=0.2em]
    \item \textbf{Klient} -- interakcja z systemem odbywa się poprzez interfejs webowy, umożliwiający rezerwację i zakup biletów online oraz zarządzanie własnymi rezerwacjami.
    
    \item \textbf{Kasjer} -- obsługa transakcji w punkcie stacjonarnym, wspomaganie klientów w procesie zakupu oraz rejestracja nowych użytkowników w systemie.
    
    \item \textbf{Menedżer Kina} -- zarządzanie lokalnym repertuarem oraz bazą klientów w obrębie pojedynczego kina.
    
    \item \textbf{Administrator Sieci} -- centralny nadzór nad całą siecią kin, zarządzanie ofertą filmową i seansami we wszystkich lokalizacjach.
\end{itemize}

\vspace{0.3cm}
\noindent Diagram ilustruje hierarchię uprawnień -- Administrator Sieci dysponuje najszerszymi możliwościami modyfikacji systemu, podczas gdy Klient ma dostęp wyłącznie do funkcji związanych z rezerwacją i przeglądaniem swoich transakcji.

\begin{figure}[h]
    \centering
    \fbox{\includegraphics[width=0.95\textwidth]{im.jpg}}
    \caption{Diagram przypadków użycia systemu sprzedaży biletów}
    \label{fig:usecase-diagram}
\end{figure}

\newpage
\section{Przypadki użycia}

\subsection{PU01. Rezerwacja biletu online}

\begin{usecase}{PU01. Rezerwacja biletu online}
\textbf{Cel:} Umożliwienie klientowi rezerwacji biletu na wybrany seans filmowy przez stronę internetową.

\vspace{0.15cm}
\noindent\textbf{Warunki wstępne:}
\begin{itemize}[nosep,topsep=0pt,leftmargin=*]
    \item Klient ma dostęp do systemu online.
    \item W systemie istnieją dostępne seanse filmowe.
    \item System przechowuje dane o salach kinowych i miejscach.
\end{itemize}

\vspace{0.15cm}
\noindent\textbf{Warunki końcowe:}
\begin{itemize}[nosep,topsep=0pt,leftmargin=*]
    \item Bilet został zarezerwowany i zablokowany w systemie dla wybranego klienta.
    \item System wysłał potwierdzenie rezerwacji klientowi e-mailem.
    \item Miejsce w sali zostało czasowo zablokowane dla transakcji klienta.
\end{itemize}

\vspace{0.15cm}
\noindent\textbf{Scenariusz główny:}
\begin{enumerate}[nosep,topsep=0pt,leftmargin=*]
    \item Klient przegląda repertuar filmowy (zawiera PU09).
    \item Klient wybiera film, datę i godzinę seansu.
    \item System wyświetla plan sali z dostępnymi miejscami.
    \item Klient wybiera jedno lub więcej miejsc w sali.
    \item System sprawdza dostępność wybranych miejsc.
    \item System tymczasowo blokuje wybrane miejsca na czas transakcji.
    \item Klient podaje dane kontaktowe (imię, nazwisko, e-mail).
    \item Klient potwierdza rezerwację i przechodzi do płatności.
    \item System zapisuje rezerwację w bazie danych.
    \item System generuje potwierdzenie rezerwacji z numerem biletu.
    \item System wysyła e-mail z potwierdzeniem i biletem elektronicznym do klienta.
\end{enumerate}

\vspace{0.15cm}
\noindent\textbf{Scenariusz alternatywny:}
\begin{itemize}[nosep,topsep=0pt,leftmargin=*]
    \item \textbf{[Krok 5]:} Jeśli wybrane miejsce zostało w międzyczasie zarezerwowane przez innego klienta, system informuje o niedostępności i prosi o wybór innego miejsca.
    \item \textbf{[Krok 6]:} Jeśli klient nie dokończy transakcji w określonym czasie (np. 10 minut), system automatycznie odblokowuje miejsca i anuluje rezerwację.
    \item \textbf{[Krok 8]:} Jeśli płatność nie powiedzie się, system anuluje rezerwację i odblokowuje miejsca.
\end{itemize}
\end{usecase}

\subsection{PU02. Anulowanie rezerwacji biletu}

\begin{usecase}{PU02. Anulowanie rezerwacji biletu}
\textbf{Cel:} Umożliwienie klientowi anulowania wcześniej zarezerwowanego biletu przed rozpoczęciem seansu.

\vspace{0.15cm}
\noindent\textbf{Warunki wstępne:}
\begin{itemize}[nosep,topsep=0pt,leftmargin=*]
    \item Klient posiada aktywną rezerwację biletu.
    \item Seans filmowy jeszcze się nie rozpoczął.
\end{itemize}

\vspace{0.15cm}
\noindent\textbf{Warunki końcowe:}
\begin{itemize}[nosep,topsep=0pt,leftmargin=*]
    \item Rezerwacja biletu została anulowana w systemie.
    \item Miejsce w sali zostało odblokowane i jest ponownie dostępne.
    \item Wykonano zwrot pieniędzy za bilet (zawiera PU03).
\end{itemize}

\vspace{0.15cm}
\noindent\textbf{Scenariusz główny:}
\begin{enumerate}[nosep,topsep=0pt,leftmargin=*]
    \item Klient loguje się do systemu online lub zwraca się do kasjera.
    \item Klient wybiera rezerwację do anulowania ze swojej historii (zawiera PU09).
    \item System wyświetla szczegóły rezerwacji (film, data, godzina, miejsca).
    \item System weryfikuje, czy seans jeszcze się nie rozpoczął.
    \item Klient potwierdza chęć anulowania rezerwacji.
    \item System usuwa rezerwację z bazy danych.
    \item System odblokowuje zarezerwowane miejsca.
    \item System inicjuje zwrot pieniędzy (zawiera PU03).
    \item System wysyła klientowi potwierdzenie anulowania rezerwacji e-mailem.
\end{enumerate}

\vspace{0.15cm}
\noindent\textbf{Scenariusz alternatywny:}
\begin{itemize}[nosep,topsep=0pt,leftmargin=*]
    \item \textbf{[Krok 4]:} Jeśli seans już się rozpoczął lub rozpoczyna się w ciągu najbliższych 30 minut, system informuje, że zwrot nie jest możliwy zgodnie z regulaminem.
    \item \textbf{[Krok 8]:} Jeśli wystąpi błąd podczas zwrotu płatności, system powiadamia klienta i zapisuje zgłoszenie do ręcznej obsługi przez administratora.
\end{itemize}
\end{usecase}

\subsection{PU03. Zwrot pieniędzy za bilet}

\begin{usecase}{PU03. Zwrot pieniędzy za bilet}
\textbf{Cel:} Realizacja zwrotu pieniędzy klientowi za anulowany bilet.

\vspace{0.15cm}
\noindent\textbf{Warunki wstępne:}
\begin{itemize}[nosep,topsep=0pt,leftmargin=*]
    \item Przypadek użycia PU02 (Anulowanie rezerwacji biletu) został zainicjowany i przekazał dane o anulowanej rezerwacji.
    \item System posiada dane o płatności za bilet (metoda płatności, kwota).
\end{itemize}

\vspace{0.15cm}
\noindent\textbf{Warunki końcowe:}
\begin{itemize}[nosep,topsep=0pt,leftmargin=*]
    \item Pieniądze zostały zwrócone klientowi na konto lub kartę, z której dokonano płatności.
    \item System zarejestrował operację zwrotu w bazie danych.
\end{itemize}

\vspace{0.15cm}
\noindent\textbf{Scenariusz główny:}
\begin{enumerate}[nosep,topsep=0pt,leftmargin=*]
    \item System pobiera dane o anulowanej rezerwacji biletu z PU02.
    \item System identyfikuje metodę płatności użytą przy zakupie biletu.
    \item System oblicza kwotę do zwrotu (zgodnie z polityką zwrotów sieci kin).
    \item System inicjuje transakcję zwrotu środków przez bramkę płatności.
    \item Bramka płatności przetwarza zwrot na konto lub kartę klienta.
    \item System zapisuje operację zwrotu w historii transakcji.
    \item System generuje potwierdzenie zwrotu dla klienta.
\end{enumerate}

\vspace{0.15cm}
\noindent\textbf{Scenariusz alternatywny:}
\begin{itemize}[nosep,topsep=0pt,leftmargin=*]
    \item \textbf{[Krok 5]:} Jeśli bramka płatności zwraca błąd, system zapisuje niepowodzenie i powiadamia klienta o konieczności kontaktu z działem obsługi klienta.
    \item \textbf{[Krok 5]:} Jeśli zwrot został zainicjowany przez kasjera (płatność gotówką), kasjer zwraca pieniądze bezpośrednio klientowi w kasie, a system rejestruje zwrot.
\end{itemize}
\end{usecase}

\subsection{PU04. Zarządzanie repertuarem filmowym sieci kin}

\begin{usecase}{PU04. Zarządzanie repertuarem filmowym sieci kin}
\textbf{Cel:} Aktualizacja i zarządzanie ofertą filmową dostępną w całej sieci kin przez Administratora Sieci.

\vspace{0.15cm}
\noindent\textbf{Warunki wstępne:}
\begin{itemize}[nosep,topsep=0pt,leftmargin=*]
    \item Administrator Sieci Kin jest zalogowany do systemu.
    \item System posiada bazę danych filmów i kin w sieci.
\end{itemize}

\vspace{0.15cm}
\noindent\textbf{Warunki końcowe:}
\begin{itemize}[nosep,topsep=0pt,leftmargin=*]
    \item Repertuar filmowy w systemie został zaktualizowany.
    \item Zmiany są widoczne dla wszystkich kin w sieci oraz dla klientów online.
\end{itemize}

\vspace{0.15cm}
\noindent\textbf{Scenariusz główny:}
\begin{enumerate}[nosep,topsep=0pt,leftmargin=*]
    \item Administrator wybiera opcję „Zarządzanie repertuarem".
    \item System wyświetla listę dostępnych filmów w bazie.
    \item Administrator wybiera operację: dodanie nowego filmu, edycję istniejącego lub usunięcie filmu.
    \item Administrator wprowadza lub modyfikuje dane filmu (tytuł, reżyser, gatunek, czas trwania, opis, plakat, klasyfikacja wiekowa).
    \item System weryfikuje poprawność wprowadzonych danych.
    \item Administrator zatwierdza zmiany.
    \item System zapisuje dane w centralnej bazie danych.
    \item System synchronizuje zmiany ze wszystkimi kinami w sieci.
\end{enumerate}

\vspace{0.15cm}
\noindent\textbf{Scenariusz alternatywny:}
\begin{itemize}[nosep,topsep=0pt,leftmargin=*]
    \item \textbf{[Krok 5]:} Jeśli wprowadzone dane są niepełne lub niepoprawne (np. brak tytułu, nieprawidłowy format czasu trwania), system wyświetla komunikat o błędzie i prosi o poprawę.
    \item \textbf{[Krok 3]:} Jeśli administrator próbuje usunąć film, dla którego zaplanowane są przyszłe seanse, system ostrzega o istniejących seansach i pyta o potwierdzenie usunięcia.
\end{itemize}
\end{usecase}

\subsection{PU05. Dodanie nowego seansu filmowego}

\begin{usecase}{PU05. Dodanie nowego seansu filmowego}
\textbf{Cel:} Wprowadzenie do systemu nowego seansu filmowego w danym kinie przez Menedżera Kina lub Administratora Sieci.

\vspace{0.15cm}
\noindent\textbf{Warunki wstępne:}
\begin{itemize}[nosep,topsep=0pt,leftmargin=*]
    \item Menedżer Kina lub Administrator Sieci jest zalogowany do systemu.
    \item W systemie istnieje film dostępny w repertuarze.
    \item W kinie istnieje sala projekcyjna z wolnymi terminami.
\end{itemize}

\vspace{0.15cm}
\noindent\textbf{Warunki końcowe:}
\begin{itemize}[nosep,topsep=0pt,leftmargin=*]
    \item Nowy seans został dodany do harmonogramu kina.
    \item Seans jest widoczny dla klientów online i kasjerów.
    \item Miejsca w sali są dostępne do rezerwacji.
\end{itemize}

\vspace{0.15cm}
\noindent\textbf{Scenariusz główny:}
\begin{enumerate}[nosep,topsep=0pt,leftmargin=*]
    \item Użytkownik wybiera opcję „Dodaj nowy seans".
    \item System wyświetla listę dostępnych filmów w repertuarze (zawiera PU04).
    \item Użytkownik wybiera film z listy.
    \item Użytkownik określa datę i godzinę rozpoczęcia seansu.
    \item System wyświetla listę dostępnych sal projekcyjnych w kinie.
    \item Użytkownik wybiera salę projekcyjną.
    \item System sprawdza dostępność sali w wybranym terminie (weryfikacja kolizji z innymi seansami).
    \item Użytkownik określa cenę biletu dla seansu (lub używa domyślnej).
    \item Użytkownik potwierdza dodanie seansu.
    \item System zapisuje nowy seans w bazie danych.
    \item System aktualizuje harmonogram kina i udostępnia seans do rezerwacji.
\end{enumerate}

\vspace{0.15cm}
\noindent\textbf{Scenariusz alternatywny:}
\begin{itemize}[nosep,topsep=0pt,leftmargin=*]
    \item \textbf{[Krok 7]:} Jeśli wybrana sala jest zajęta w podanym terminie (kolizja z innym seansem), system wyświetla komunikat o konflikcie i prosi o wybór innej sali lub innego terminu.
    \item \textbf{[Krok 4]:} Jeśli użytkownik poda datę w przeszłości, system wyświetla błąd i prosi o podanie przyszłej daty.
\end{itemize}
\end{usecase}

\subsection{PU06. Edytowanie istniejącego seansu}

\begin{usecase}{PU06. Edytowanie istniejącego seansu}
\textbf{Cel:} Umożliwienie Menedżerowi Kina lub Administratorowi Sieci modyfikacji parametrów istniejącego seansu filmowego.

\vspace{0.15cm}
\noindent\textbf{Warunki wstępne:}
\begin{itemize}[nosep,topsep=0pt,leftmargin=*]
    \item Menedżer Kina lub Administrator Sieci jest zalogowany do systemu.
    \item W systemie istnieje seans, który wymaga modyfikacji.
\end{itemize}

\vspace{0.15cm}
\noindent\textbf{Warunki końcowe:}
\begin{itemize}[nosep,topsep=0pt,leftmargin=*]
    \item Dane seansu zostały zaktualizowane w systemie.
    \item Zmiany są widoczne dla klientów i kasjerów.
    \item Istniejące rezerwacje pozostają ważne (jeśli zmiany nie są krytyczne) lub klienci zostają powiadomieni o zmianach.
\end{itemize}

\vspace{0.15cm}
\noindent\textbf{Scenariusz główny:}
\begin{enumerate}[nosep,topsep=0pt,leftmargin=*]
    \item Użytkownik wybiera opcję „Zarządzanie seansami".
    \item System wyświetla listę wszystkich seansów w kinie.
    \item Użytkownik wybiera seans do edycji.
    \item System wyświetla szczegóły seansu (film, data, godzina, sala, cena).
    \item Użytkownik modyfikuje wybrane parametry (datę, godzinę, salę lub cenę).
    \item System weryfikuje poprawność nowych danych (np. dostępność sali w nowym terminie).
    \item Użytkownik zatwierdza zmiany.
    \item System zapisuje zaktualizowane dane seansu.
    \item System aktualizuje harmonogram kina.
\end{enumerate}

\vspace{0.15cm}
\noindent\textbf{Scenariusz alternatywny:}
\begin{itemize}[nosep,topsep=0pt,leftmargin=*]
    \item \textbf{[Krok 6]:} Jeśli nowa godzina lub sala koliduje z innym seansem, system wyświetla komunikat o konflikcie i prosi o wprowadzenie innych danych.
    \item \textbf{[Krok 5]:} Jeśli dla seansu istnieją już rezerwacje, a zmiana dotyczy daty, godziny lub sali, system ostrzega użytkownika i wymaga dodatkowego potwierdzenia. Po potwierdzeniu system wysyła powiadomienia e-mail do klientów z istniejącymi rezerwacjami.
    \item \textbf{[Krok 5]:} Jeśli użytkownik próbuje zmienić datę na przeszłą, system wyświetla błąd i odrzuca zmianę.
\end{itemize}
\end{usecase}

\subsection{PU07. Rejestracja nowego klienta w systemie}

\begin{usecase}{PU07. Rejestracja nowego klienta w systemie}
\textbf{Cel:} Utworzenie konta dla nowego klienta w systemie, umożliwiającego rezerwację biletów online i zarządzanie własnymi rezerwacjami.

\vspace{0.15cm}
\noindent\textbf{Warunki wstępne:}
\begin{itemize}[nosep,topsep=0pt,leftmargin=*]
    \item Klient nie posiada jeszcze konta w systemie.
    \item Kasjer jest zalogowany (w przypadku rejestracji w kasie) lub system rejestracji online jest dostępny.
\end{itemize}

\vspace{0.15cm}
\noindent\textbf{Warunki końcowe:}
\begin{itemize}[nosep,topsep=0pt,leftmargin=*]
    \item Nowy klient został dodany do bazy danych systemu.
    \item Klient otrzymał dane dostępowe do systemu (login, hasło).
    \item Klient może logować się i korzystać z funkcji rezerwacji biletów online.
\end{itemize}

\vspace{0.15cm}
\noindent\textbf{Scenariusz główny:}
\begin{enumerate}[nosep,topsep=0pt,leftmargin=*]
    \item Klient wybiera opcję „Rejestracja" na stronie internetowej lub zgłasza się do kasjera.
    \item Klient lub kasjer wprowadza dane osobowe: imię, nazwisko, adres e-mail, numer telefonu.
    \item Klient tworzy hasło (w przypadku rejestracji online) lub kasjer generuje tymczasowe hasło.
    \item System weryfikuje unikalność adresu e-mail w bazie danych.
    \item System sprawdza poprawność formatu wprowadzonych danych (e-mail, telefon).
    \item Klient akceptuje regulamin i politykę prywatności (zgoda RODO).
    \item System zapisuje dane klienta w bazie danych.
    \item System wysyła e-mail aktywacyjny z linkiem potwierdzającym rejestrację.
    \item Klient potwierdza rejestrację poprzez kliknięcie w link w e-mailu.
    \item System aktywuje konto klienta.
\end{enumerate}

\vspace{0.15cm}
\noindent\textbf{Scenariusz alternatywny:}
\begin{itemize}[nosep,topsep=0pt,leftmargin=*]
    \item \textbf{[Krok 4]:} Jeśli adres e-mail już istnieje w systemie, system informuje o istniejącym koncie i proponuje opcję przypomnienia hasła.
    \item \textbf{[Krok 5]:} Jeśli format danych jest niepoprawny (np. nieprawidłowy e-mail), system wyświetla komunikat o błędzie i prosi o poprawę.
    \item \textbf{[Krok 9]:} Jeśli klient nie aktywuje konta w ciągu 24 godzin, system wysyła przypomnienie. Po 7 dniach bez aktywacji, konto zostaje usunięte.
\end{itemize}
\end{usecase}

\subsection{PU08. Zarządzanie danymi klientów}

\begin{usecase}{PU08. Zarządzanie danymi klientów}
\textbf{Cel:} Umożliwienie Menedżerowi Kina edycji i zarządzania danymi klientów w bazie systemu.

\vspace{0.15cm}
\noindent\textbf{Warunki wstępne:}
\begin{itemize}[nosep,topsep=0pt,leftmargin=*]
    \item Menedżer Kina jest zalogowany do systemu.
    \item W systemie istnieje baza danych zarejestrowanych klientów.
\end{itemize}

\vspace{0.15cm}
\noindent\textbf{Warunki końcowe:}
\begin{itemize}[nosep,topsep=0pt,leftmargin=*]
    \item Dane klienta zostały zaktualizowane w systemie.
    \item Zmiany są zapisane w bazie danych zgodnie z wymogami RODO.
\end{itemize}

\vspace{0.15cm}
\noindent\textbf{Scenariusz główny:}
\begin{enumerate}[nosep,topsep=0pt,leftmargin=*]
    \item Menedżer wybiera opcję „Zarządzanie klientami".
    \item System wyświetla listę zarejestrowanych klientów z możliwością wyszukiwania.
    \item Menedżer wyszukuje klienta (po imieniu, nazwisku lub e-mailu).
    \item Menedżer wybiera klienta z listy.
    \item System wyświetla szczegółowe dane klienta (imię, nazwisko, e-mail, telefon, status konta).
    \item Menedżer edytuje wybrane dane kontaktowe lub status konta klienta.
    \item System weryfikuje poprawność wprowadzonych danych.
    \item Menedżer zatwierdza zmiany.
    \item System zapisuje zaktualizowane dane klienta w bazie.
    \item System rejestruje operację zgodnie z wymogami RODO (kto, kiedy i co zmienił).
\end{enumerate}

\vspace{0.15cm}
\noindent\textbf{Scenariusz alternatywny:}
\begin{itemize}[nosep,topsep=0pt,leftmargin=*]
    \item \textbf{[Krok 7]:} Jeśli wprowadzone dane są niepoprawne (np. nieprawidłowy format e-maila), system wyświetla komunikat o błędzie i prosi o poprawę.
    \item \textbf{[Krok 6]:} Jeśli menedżer próbuje dezaktywować konto klienta, który ma aktywne rezerwacje, system ostrzega o istniejących rezerwacjach i wymaga dodatkowego potwierdzenia.
    \item \textbf{[Krok 3]:} Jeśli klient nie zostanie znaleziony, system informuje o braku wyników i oferuje opcję dodania nowego klienta (zawiera PU07).
\end{itemize}
\end{usecase}

\subsection{PU09. Przeglądanie historii rezerwacji i repertuaru}

\begin{usecase}{PU09. Przeglądanie historii rezerwacji i repertuaru}
\textbf{Cel:} Umożliwienie klientowi przeglądania dostępnego repertuaru filmowego oraz historii własnych rezerwacji biletów.

\vspace{0.15cm}
\noindent\textbf{Warunki wstępne:}
\begin{itemize}[nosep,topsep=0pt,leftmargin=*]
    \item Klient ma dostęp do systemu online (może być zalogowany lub przeglądać repertuar bez logowania).
    \item System posiada dane o seansach filmowych i rezerwacjach klienta.
\end{itemize}

\vspace{0.15cm}
\noindent\textbf{Warunki końcowe:}
\begin{itemize}[nosep,topsep=0pt,leftmargin=*]
    \item Klient wyświetlił repertuar lub historię swoich rezerwacji.
    \item Klient uzyskał potrzebne informacje o filmach, seansach lub swoich biletach.
\end{itemize}

\vspace{0.15cm}
\noindent\textbf{Scenariusz główny — Przeglądanie repertuaru:}
\begin{enumerate}[nosep,topsep=0pt,leftmargin=*]
    \item Klient wchodzi na stronę internetową systemu.
    \item System wyświetla aktualny repertuar filmowy dla wszystkich kin w sieci.
    \item Klient może filtrować repertuar według: kina, daty, gatunku filmu, godziny seansu.
    \item System wyświetla listę dostępnych seansów zgodnie z wybranymi kryteriami.
    \item Klient wybiera konkretny seans, aby zobaczyć szczegóły (opis filmu, obsada, czas trwania, dostępność miejsc).
    \item Klient może przejść do rezerwacji biletu (rozszerza przez PU01).
\end{enumerate}

\vspace{0.15cm}
\noindent\textbf{Scenariusz główny — Przeglądanie historii rezerwacji:}
\begin{enumerate}[nosep,topsep=0pt,leftmargin=*]
    \item Klient loguje się do systemu.
    \item Klient wybiera opcję „Moje rezerwacje" lub „Historia".
    \item System pobiera dane o rezerwacjach klienta z bazy danych.
    \item System wyświetla listę wszystkich rezerwacji (przeszłych i przyszłych) z informacjami: data, film, godzina, sala, miejsca, status.
    \item Klient może wybrać konkretną rezerwację, aby zobaczyć szczegóły i pobrać bilet.
    \item Klient może anulować przyszłą rezerwację (rozszerza przez PU02).
\end{enumerate}

\vspace{0.15cm}
\noindent\textbf{Scenariusz alternatywny:}
\begin{itemize}[nosep,topsep=0pt,leftmargin=*]
    \item \textbf{[Przeglądanie historii, krok 4]:} Jeśli klient nie ma żadnych rezerwacji, system wyświetla komunikat „Nie masz jeszcze żadnych rezerwacji" i oferuje przejście do repertuaru.
    \item \textbf{[Przeglądanie repertuaru, krok 4]:} Jeśli brak seansów spełniających wybrane kryteria filtrowania, system informuje o braku wyników i sugeruje zmianę kryteriów.
\end{itemize}
\end{usecase}

\end{document}
